\documentclass[12pt]{article}

\begin{document}

\author{Roy Triesschijn, Roan Kattouw and Jan Paul Posma}
\date{\today}
\title{Transparent distributed caching over REST}

\maketitle

\section*{Context}
For websites and web applications it is important that pages are served quickly. Usually a database is used to store information, and it can take a little while to retrieve this information. Then this information has to be processed and inserted in a layout, which can also take time. In order to make the serving often accessed information more performant, caching solutions can be used. A possibility would be to use a local cache in memory, but the amount of information that can be stored in such a cache is limited. Therefore, a good option for a caching layer instead or on top of this local cache, would be a distributed cache. Such a cache is programmed for performance and can typically only function as a key-value store.

A popular example of a distributed cache is \emph{memcached}. Memcached uses a server application which is really fast but does not have any logic functions at all. The client has to select which server it approaches for retrieving and storing a value for a certain key. This is usually done by a hashing function, but depends on the implementation of the client library and might therefore be inconsistent when using multiple implementations or programming languages sharing the same cached data.

Another problem of memcached is that it uses its own API. While there currently are implementations for most programming languages, it can also be convenient to use standard protocols such as REST over HTTP, while not having to worry about which server to select.

\section*{Goal}
This, indeed, is the main objective of our distributed caching system. To provide a convenient cache that is easy to set up and easy to access. While it should be performant as it is a caching server, this is not the main objective, as this is mostly an academic exercise. It should provide a transparent API, where any user can call any of the caching server, and the server should then silently pass on the request to the correct server.

\end{document}
